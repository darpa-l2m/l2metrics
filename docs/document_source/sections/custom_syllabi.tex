\chapter{Creating Custom Syllabi}\label{ch:custom_syllabi}

With the Core Capabilties in mind, we move on to intentional design of the syllabi. As a reminder, the calculation of these metrics seeks to answer a question that may differ from traditional reinforcement learning evaluation criteria, and thus, we enforce a semi-rigid structure to these evaluation tasks.

\section{Continual Learning}    
\textit{Syllabus Structure Requirements}\\
\noindent\rule{8cm}{0.4pt}
\flushleft 1. This syllabus consists of a single task with parametric variations\\
\textit{Can involve interpolation of parameters, noisy parameters, etc. but has only one task \\}


\flushleft 2. Each phase consists of a training and optional but recommended evaluation block\\
\textit{Training block is training on one or more parametric variations \\}
\noindent\rule{8cm}{0.4pt}

\flushleft\textit{Metric Evaluation Questions}\\[0.1in]

Note that some of these are computed as a basis for comparison for future eval blocks, etc.\\[0.1in]

\textbf{Question:} What level of performance does the agent achieve during training?\\
\textit{Metric:} Saturation Value\\[0.1in]


\textbf{Question:} How quickly does it achieve this saturation value?\\
\textit{Metric:} Time to Saturation\\[0.1in]

\textbf{Question:} How quickly does an agent learn during training?\\
\textit{Metric:} Normalized Integral of Reward/Time\\[0.1in]


\textbf{Question:} Can the agent adjust to changes in the environment? How quickly does it recover?\\
\textit{Metric:} Recovery Time\\[0.1in]


\textbf{Question:} Can the agent maintain performance on previously learned parameters after being trained with new ones?\\
\textit{Metric:} Performance Maintenance\\[0.1in]


\textbf{Question:} How does the lifelong learning agent's performance compare to a traditional agent?\\
\textit{Metric:} Performance relative to STE \\[0.2in]
            
            

After these metrics are computed, the agent will receive a score for  each of these metrics for a Continual Learning syllabus, which will be aggregated to form its Continual Learning score. For now, an average of the scores achieved is reported in each phase/block, but more sophisticated methods may replace these in the future. See below for more details; more information is on its way.
    

\section{Adapting to New Tasks}


\subsection*{Formal Metric Definitions}

\textbf{1. Saturation Value}\\
Purpose: The saturation value is computed to quantify the maximum maintained value by the agent.\\
Calculated by: Since multiple rewards may be logged per episode, the mean is taken per episode. Then, the max of the rolling average is taken of the mean reward per episode with a smoothing parameter, s (default, 0.1)\\
Compared to: Future or single task expert saturation values of the same task\\[0.1in]         


\textbf{2. Time to Saturation}\\
Purpose: Time to saturation is used to quantify how quickly, in number of episodes, the agent took to achieve the saturation value computed above.\\
Calculated by: The first time the saturation value (or above) is seen, that episode number is recorded\\
Compared to: Future times to saturation of the same task\\[0.1in]


\textbf{3. Normalized Integral of Reward/Time}\\
Purpose: Taking the Integral of Accumulated Reward over Time allows for a more robust comparison of the time to learn a particular task, taking into account both the shape and saturation of the learning for future comparison. Has limitations; must be normalized by length; only training phases can be compared to each other \\
Calculated by: Integrating reward over time, then dividing by the number of episodes used to accumulate the reward\\
Compared to: Future training instances of this metric\\[0.1in]


\textbf{4. Recovery Time}\\
Purpose: Recovery time is calculated to determine how quickly (if at all) an agent can "bounce back" after a change is introduced to its environment\\
Calculated by: After some training phase achieves a saturation value, determine how many episodes, if any, it takes for the agent to regain the same performance on the same task\\
Compared to: An agent's recovery time is comparable across tasks\\[0.1in]


\textbf{5. Performance Maintenance on Test Sets}\\
Purpose: Performance maintenance on test sets is calculated to determine whether an agent catastrophically forgets a previously learned task\\
Calculated by: Comparing all computed metrics on the train set (saturation values, time to saturation, etc) on the test set and computing the difference in performance\\[0.1in]


\textbf{6. Performance relative to STE (training)}\\
Purpose: STE Relative performance assesses whether a lifelong learner outperforms a traditional learner.\\
Calculated by: Normalizing metrics computed on the lifelong learner by the same metrics computed on the traditional learner\\
