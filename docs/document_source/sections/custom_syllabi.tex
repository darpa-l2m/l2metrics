\chapter{Creating Custom Syllabi}\label{ch:custom_syllabi}

As mentioned in previous sections, a semi-rigid structure is required for syllabi in the Metrics Framework. 

\subsection*{Phase Annotations}

Syllabi used to generate the log files \textbf{must} include annotations with phase information in the following format: <phase\_number>.<phase\_type>, where phase\_number is an increasing integer beginning with 1, and phase\_type can be either "train" or "test". No other annotations are currently supported. Pleaase note that without these annotations, the Metrics Framework will not properly compute the Proposed Metrics. \\[0.1in]


\textit{Valid Phase annotation format example}:\\[0.1in]
\begin{small}
\begin{verbatim}
Train: {"$phase":  "1.train"}, {"$phase":  "2.train"}, ..., {"$phase":  "n.train"}
Test: {"$phase":  "1.test"}, {"$phase":  "2.test"}, ..., {"$phase":  "n.test"}
\end{verbatim}
\end{small}

\subsection*{Train/Test Structure}

\begin{enumerate}
\item Expected
\begin{itemize}
\item Train phases (consisting of 1 or more blocks) followed by Test phases (consisting of 1 or more blocks)
\end{itemize}

\item Required
\begin{itemize}
\item First blocks must be Train blocks
\end{itemize}

\item Disallowed
\begin{itemize}
\item Continual Learning Syllabi are not allowed to exercise more than one task
\end{itemize}
\end{enumerate}

\section{Continual Learning}
\textit{Structure}\\[0.1in]

The purpose of this type of syllabus is to assess whether the agent can adjust to changes in the environment and maintain performance on the previous parameters when new ones are introduced. There is only one type of Continual Learning syllabus.

\begin{itemize}
\item A Continual Learning syllabus consists of a \textbf{single} task with parametric variations\\
\textit{Can involve interpolation of parameters, noisy parameters, etc. but has only one task \\}


\item Each phase in a Continual Learning syllabus consists of a training and optional but recommended testing block\\
\textit{Training block is training on one or more parametric variations}\\[0.1in]
\end{itemize}

\section{Adapting to New Tasks}
\textit{Structure}\\[0.1in]

The purpose of this type of syllabus is to assess whether the agent can adjust to changes in the environment and maintain performance on the previous tasks when new ones are introduced. Consists of multiple tasks, generally without parametric variations except in Subtype C. Testing phase is mandatory. There are three subtypes of an Adapting to New Tasks syllabus.

\begin{enumerate}
    \item \textbf{Subtype A:} Assess Resistance to Catastrophic Interference\\
        Consists of multiple tasks with testing blocks to compute performance maintenance after learning new tasks. Parametric variations within a task are not exercised.\\
                    
    \item \textbf{Subtype B:} Assess Basic skill transfer\\
        Consists of multiple tasks with appropriate testing blocks to compute transfer matrix. Parametric variations within a task are not exercised.\\
            
    \item \textbf{Subtype C:} Assess Parametric skill transfer\\
        Consists of parametric variation in Task 1 and appropriate testing block to evaluate parametric skill tranfers to task 2.\\
        %L2Arcade Example: vary paddle width in Pong; does this transfer to varying paddle width in Breakout? \\
        \textit{This can be formulated as both forward and reverse transfer.}\\
\end{enumerate}

\iffalse

Subtype A
Question: Can the agent learn a new task without forgetting the old task?\\
                Metric: Recovery time (relative to previous task baseline)\\
                    Calculated within and across training blocks \\
                    After a new task is introduced, does performance fall? Does it recover to the previous level? How quickly does it recover? \\
                Metric: Performance (reward statistics) on previously learned tasks\\
                    Calculated within testing blocks\\
                    Note that this is not to assess forward or reverse transfer; transfer is assessed in ANT Subtype B\\
Subtype B   
Question: Does the agent transfer knowledge from a learned task to a new task?\\
            Metric: Recovery time (relative to previous task baseline)\\
                Calculated within and across training blocks \\
                After a new task is introduced, does performance fall? Does it recover to the previous level? How quickly does it recover? \\
            Metric: Performance (reward statistics) on previously learned tasks\\
                Calculated within testing blocks\\
                Note that this is not to assess forward or reverse transfer; transfer is assessed in ANT Subtype B\\
        Metrics are:\\
            Time (number of episodes) to achieve threshold performance on Task 2 after being trained on Task 1\\
            Transfer matrix\\        
Subtype C           
        Questions\\
            Transfer of general ability: does parametric variation in Pong improve Breakout time-to-learn at all\\
            Transfer of continual learning: Does parametric variation (continual learning) in Pong improve continual learning in Breakout?\\
        Metrics are:\\
            Amount of time required to learn vs from scratch \\
\fi
